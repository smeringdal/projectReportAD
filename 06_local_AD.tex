\chapter{Local Automatic Differentiation}
\label{ch:LocalAD}
This chapter will consider a different approach on how to use AD to solve PDEs. When solving PDEs using a finite element method each cell will only depend on the neighbouring cells. In the \texttt{FAD} and \texttt{CJAD} implementations, the dependencies are stored in the discrete gradient and divergence operators. The implementation of \texttt{FAD} and \texttt{CJAD} calculates the residual function value and corresponding Jacobian for the whole grid simultaneously by having a vector to store the values and sparse matrices for the Jacobians. Hence when the discrete gradient and divergence operators are used in the calculation of the residual function in \autoref{ch:FlowSolver}, the Jacobian are automatically obtained with the structure seen in \autoref{fig:flowSolverJacobian}. However, this structure is known from the grid properties given in the \texttt{G} variable. So instead of calculating the residual function for all cells at once, the new approach takes one cell at the time and sums up the contributions from each neighbour. This is done for each cell until all the residual functions for each cell is calculated. This new approach is called local AD and the method is based on the same idea as how AD is done in OPM \emph{\citep{opm}}. \textbf{Since OPM is written in C and C++ it is interesting to see if you can write similar type of code in Julia and obtain the same computational efficiency.} 

\section{Implementation}
To get a better understanding of how Local AD works, it is best to look at how it is implemented. Like for \texttt{FAD} and \texttt{CJAD} I have implemented local AD using a struct I have called \texttt{LAD}.








\todo[inline]{Se på parallellisering for lokal AD?}